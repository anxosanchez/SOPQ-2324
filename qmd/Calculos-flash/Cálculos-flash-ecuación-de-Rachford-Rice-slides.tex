% Options for packages loaded elsewhere
\PassOptionsToPackage{unicode}{hyperref}
\PassOptionsToPackage{hyphens}{url}
\PassOptionsToPackage{dvipsnames,svgnames,x11names}{xcolor}
%
\documentclass[
  letterpaper,
  DIV=11,
  numbers=noendperiod]{scrartcl}

\usepackage{amsmath,amssymb}
\usepackage{iftex}
\ifPDFTeX
  \usepackage[T1]{fontenc}
  \usepackage[utf8]{inputenc}
  \usepackage{textcomp} % provide euro and other symbols
\else % if luatex or xetex
  \usepackage{unicode-math}
  \defaultfontfeatures{Scale=MatchLowercase}
  \defaultfontfeatures[\rmfamily]{Ligatures=TeX,Scale=1}
\fi
\usepackage{lmodern}
\ifPDFTeX\else  
    % xetex/luatex font selection
\fi
% Use upquote if available, for straight quotes in verbatim environments
\IfFileExists{upquote.sty}{\usepackage{upquote}}{}
\IfFileExists{microtype.sty}{% use microtype if available
  \usepackage[]{microtype}
  \UseMicrotypeSet[protrusion]{basicmath} % disable protrusion for tt fonts
}{}
\makeatletter
\@ifundefined{KOMAClassName}{% if non-KOMA class
  \IfFileExists{parskip.sty}{%
    \usepackage{parskip}
  }{% else
    \setlength{\parindent}{0pt}
    \setlength{\parskip}{6pt plus 2pt minus 1pt}}
}{% if KOMA class
  \KOMAoptions{parskip=half}}
\makeatother
\usepackage{xcolor}
\setlength{\emergencystretch}{3em} % prevent overfull lines
\setcounter{secnumdepth}{-\maxdimen} % remove section numbering
% Make \paragraph and \subparagraph free-standing
\ifx\paragraph\undefined\else
  \let\oldparagraph\paragraph
  \renewcommand{\paragraph}[1]{\oldparagraph{#1}\mbox{}}
\fi
\ifx\subparagraph\undefined\else
  \let\oldsubparagraph\subparagraph
  \renewcommand{\subparagraph}[1]{\oldsubparagraph{#1}\mbox{}}
\fi

\usepackage{color}
\usepackage{fancyvrb}
\newcommand{\VerbBar}{|}
\newcommand{\VERB}{\Verb[commandchars=\\\{\}]}
\DefineVerbatimEnvironment{Highlighting}{Verbatim}{commandchars=\\\{\}}
% Add ',fontsize=\small' for more characters per line
\usepackage{framed}
\definecolor{shadecolor}{RGB}{241,243,245}
\newenvironment{Shaded}{\begin{snugshade}}{\end{snugshade}}
\newcommand{\AlertTok}[1]{\textcolor[rgb]{0.68,0.00,0.00}{#1}}
\newcommand{\AnnotationTok}[1]{\textcolor[rgb]{0.37,0.37,0.37}{#1}}
\newcommand{\AttributeTok}[1]{\textcolor[rgb]{0.40,0.45,0.13}{#1}}
\newcommand{\BaseNTok}[1]{\textcolor[rgb]{0.68,0.00,0.00}{#1}}
\newcommand{\BuiltInTok}[1]{\textcolor[rgb]{0.00,0.23,0.31}{#1}}
\newcommand{\CharTok}[1]{\textcolor[rgb]{0.13,0.47,0.30}{#1}}
\newcommand{\CommentTok}[1]{\textcolor[rgb]{0.37,0.37,0.37}{#1}}
\newcommand{\CommentVarTok}[1]{\textcolor[rgb]{0.37,0.37,0.37}{\textit{#1}}}
\newcommand{\ConstantTok}[1]{\textcolor[rgb]{0.56,0.35,0.01}{#1}}
\newcommand{\ControlFlowTok}[1]{\textcolor[rgb]{0.00,0.23,0.31}{#1}}
\newcommand{\DataTypeTok}[1]{\textcolor[rgb]{0.68,0.00,0.00}{#1}}
\newcommand{\DecValTok}[1]{\textcolor[rgb]{0.68,0.00,0.00}{#1}}
\newcommand{\DocumentationTok}[1]{\textcolor[rgb]{0.37,0.37,0.37}{\textit{#1}}}
\newcommand{\ErrorTok}[1]{\textcolor[rgb]{0.68,0.00,0.00}{#1}}
\newcommand{\ExtensionTok}[1]{\textcolor[rgb]{0.00,0.23,0.31}{#1}}
\newcommand{\FloatTok}[1]{\textcolor[rgb]{0.68,0.00,0.00}{#1}}
\newcommand{\FunctionTok}[1]{\textcolor[rgb]{0.28,0.35,0.67}{#1}}
\newcommand{\ImportTok}[1]{\textcolor[rgb]{0.00,0.46,0.62}{#1}}
\newcommand{\InformationTok}[1]{\textcolor[rgb]{0.37,0.37,0.37}{#1}}
\newcommand{\KeywordTok}[1]{\textcolor[rgb]{0.00,0.23,0.31}{#1}}
\newcommand{\NormalTok}[1]{\textcolor[rgb]{0.00,0.23,0.31}{#1}}
\newcommand{\OperatorTok}[1]{\textcolor[rgb]{0.37,0.37,0.37}{#1}}
\newcommand{\OtherTok}[1]{\textcolor[rgb]{0.00,0.23,0.31}{#1}}
\newcommand{\PreprocessorTok}[1]{\textcolor[rgb]{0.68,0.00,0.00}{#1}}
\newcommand{\RegionMarkerTok}[1]{\textcolor[rgb]{0.00,0.23,0.31}{#1}}
\newcommand{\SpecialCharTok}[1]{\textcolor[rgb]{0.37,0.37,0.37}{#1}}
\newcommand{\SpecialStringTok}[1]{\textcolor[rgb]{0.13,0.47,0.30}{#1}}
\newcommand{\StringTok}[1]{\textcolor[rgb]{0.13,0.47,0.30}{#1}}
\newcommand{\VariableTok}[1]{\textcolor[rgb]{0.07,0.07,0.07}{#1}}
\newcommand{\VerbatimStringTok}[1]{\textcolor[rgb]{0.13,0.47,0.30}{#1}}
\newcommand{\WarningTok}[1]{\textcolor[rgb]{0.37,0.37,0.37}{\textit{#1}}}

\providecommand{\tightlist}{%
  \setlength{\itemsep}{0pt}\setlength{\parskip}{0pt}}\usepackage{longtable,booktabs,array}
\usepackage{calc} % for calculating minipage widths
% Correct order of tables after \paragraph or \subparagraph
\usepackage{etoolbox}
\makeatletter
\patchcmd\longtable{\par}{\if@noskipsec\mbox{}\fi\par}{}{}
\makeatother
% Allow footnotes in longtable head/foot
\IfFileExists{footnotehyper.sty}{\usepackage{footnotehyper}}{\usepackage{footnote}}
\makesavenoteenv{longtable}
\usepackage{graphicx}
\makeatletter
\def\maxwidth{\ifdim\Gin@nat@width>\linewidth\linewidth\else\Gin@nat@width\fi}
\def\maxheight{\ifdim\Gin@nat@height>\textheight\textheight\else\Gin@nat@height\fi}
\makeatother
% Scale images if necessary, so that they will not overflow the page
% margins by default, and it is still possible to overwrite the defaults
% using explicit options in \includegraphics[width, height, ...]{}
\setkeys{Gin}{width=\maxwidth,height=\maxheight,keepaspectratio}
% Set default figure placement to htbp
\makeatletter
\def\fps@figure{htbp}
\makeatother

\KOMAoption{captions}{tableheading}
\makeatletter
\@ifpackageloaded{caption}{}{\usepackage{caption}}
\AtBeginDocument{%
\ifdefined\contentsname
  \renewcommand*\contentsname{Table of contents}
\else
  \newcommand\contentsname{Table of contents}
\fi
\ifdefined\listfigurename
  \renewcommand*\listfigurename{List of Figures}
\else
  \newcommand\listfigurename{List of Figures}
\fi
\ifdefined\listtablename
  \renewcommand*\listtablename{List of Tables}
\else
  \newcommand\listtablename{List of Tables}
\fi
\ifdefined\figurename
  \renewcommand*\figurename{Figure}
\else
  \newcommand\figurename{Figure}
\fi
\ifdefined\tablename
  \renewcommand*\tablename{Table}
\else
  \newcommand\tablename{Table}
\fi
}
\@ifpackageloaded{float}{}{\usepackage{float}}
\floatstyle{ruled}
\@ifundefined{c@chapter}{\newfloat{codelisting}{h}{lop}}{\newfloat{codelisting}{h}{lop}[chapter]}
\floatname{codelisting}{Listing}
\newcommand*\listoflistings{\listof{codelisting}{List of Listings}}
\makeatother
\makeatletter
\makeatother
\makeatletter
\@ifpackageloaded{caption}{}{\usepackage{caption}}
\@ifpackageloaded{subcaption}{}{\usepackage{subcaption}}
\makeatother
\ifLuaTeX
  \usepackage{selnolig}  % disable illegal ligatures
\fi
\usepackage{bookmark}

\IfFileExists{xurl.sty}{\usepackage{xurl}}{} % add URL line breaks if available
\urlstyle{same} % disable monospaced font for URLs
\hypersetup{
  pdftitle={Cálculos pTflash, burbulla e orballo},
  pdfauthor={Anxo Sánchez},
  colorlinks=true,
  linkcolor={blue},
  filecolor={Maroon},
  citecolor={Blue},
  urlcolor={Blue},
  pdfcreator={LaTeX via pandoc}}

\title{Cálculos pTflash, burbulla e orballo}
\author{Anxo Sánchez}
\date{}

\begin{document}
\maketitle

\subsection{Cálculos flash}\label{cuxe1lculos-flash}

Figura 1. Tanque flash.

\begin{center}\rule{0.5\linewidth}{0.5pt}\end{center}

\subsection{Aplicacións}\label{aplicaciuxf3ns}

Os cálculos de flash úsanse para procesos de equilibrio vapor/líquido
(VLE). Un proceso típico que require cálculos de flash , é aquel no que
fluxo de alimentación (F) se separa nun produto vapor (V) e outro
líquido (L) como na figura anterior.

\subsection{Tipos}\label{tipos}

En principio, os cálculos de flash son sinxelos e implican combinar as
ecuacións de VLE cos balances de materia dos compoñentes e, nalgúns
casos, o balance de enerxía . Algúns cálculos de flash son:

\begin{enumerate}
\def\labelenumi{\arabic{enumi}.}
\tightlist
\item
  Punto de burbulla a unha \(T\) dada (fácil)\\
\item
  Punto de burbulla a unha \(p\) dada (hase de iterar en T)
\item
  Punto de orballo a unha \(T\) dada (fácil)
\item
  Punto de orballo a unha \(p\) dada (hase de iterar en T )
\item
  Flash a \(p\) e \(T\) (relativamente fácil)
\item
  flash dadas \(p\) e \(H\) (flash estándar, por exemplo, para un flash
  de tanque despois de unha válvula)
\item
  Flash dadas \(p\) e \(S\) (por exemplo, para unha turbina de
  condensación)
\item
  Flash dadas \(U\) e \(V\) (por exemplo, para a simulación dinámica
  dunha batería de flash adiabáticos)
\end{enumerate}

Os últimos tres son un pouco máis complicadas e esixen o cómputo de
relacións de equilibrio e balances de enerxía \(H\) , \(S\) , etc. A
continuación se ilustran algúns cálculos flash. En todos eles se asumime
que o VLE ven dado polos valores de \(K\), que é:

\[
y_i = K_i \cdot x_i
\]

onde \(y_i\) é a fraccións molares de cada compoñente \(i\) na fase
vapor e \(x_i\) as fraccións molares de cada compoñente \(i\) na fase
líquida. En xeral, valor de \(K\) depende da temperatura \(T\) , a
presión, \(p\) e a composición (ambas, \(x_i\) e \(y_i\)). Supoñemos
mestura idal, e usamos a \textbf{Lei de Raoult}. Neste caso, \(K_i\)
depende de \(T\) e \(p\) somentes:

\[
\texttt{Lei de Raoult:} \;\;\; K_i = \frac {p^{saturación}_i (T)} {p}
\]

A presión de saturación, \(p^{sat} (T)\) utilizando os parámetros da
\textbf{ecuación de Antoine}.

\subsubsection{Cálculos do punto de
burbulla}\label{cuxe1lculos-do-punto-de-burbulla}

Consideremos primeiro cálculos de puntos de burbulla. Neste caso dáse a
composición en fase líquida \(x_i\) (corresponde ao caso en que \(V\) é
moi pequena (\(V \geq 20\)) e \(x_i = z_i\) na figura). O punto de
burbulla dun líquido é o punto no que o líquido xusto comeza a
evaporarse (ferver), que é, cando a primeira burbulla de vapor se forma.
Dada unha temperatura constante, debese diminuír a presión ata que se
forma a primeira burbulla. Dada unha presión constante, débese aumentar
a temperatura ata que se forma a primeira burbulla. En ambos os casos,
esto correspondese con axustar \(T\) ou \(p\) ata que a suma de das
fraccións de vpor é a unidade, \$ \sum y\_i = 1\$ ou:

\[
\sum_i K_i x_i = 1
\]

onde se coñece \(x_i\) . Para o caso ideal onde se cumpre a \emph{lei de
Raoult} isto dá:

\[
\sum_i x_i p_i^{sat} (T) = p
\]

\subsubsection{\texorpdfstring{Exemplo. Punto de burbulla á temperatura
\(T\)
dada.}{Exemplo. Punto de burbulla á temperatura T dada.}}\label{exemplo.-punto-de-burbulla-uxe1-temperatura-t-dada.}

Unha mestura líquida contén 50\% de pentano, 30\% de hexano e 20\% de
ciclohexano (en moles), é dicir,
\(x_{pentano} = 0.5; x_{hexano} = 0.3 e x_{ciclohexano} = 0.2\). A
\(T = 400 K\) se baixa a presión gradulmente. Cál é a presión de
burbulla e a composición do primeiro vapor que condensa?. Supoñer unha
mestura líquida ideal e gas ideal (\textbf{lei de Raoult}).

As fraccións molares calcúlanse a partires da \emph{ecuación de Dalton}:

\[
p_{total} = \sum_i p_i
\]

logo:

\[
p_i = p_{total}* x_i
\]

Por tanto:

\[
x_i = \frac{p_i}{p_{total}}
\]

\subsubsection{Exemplo Punto de burbula a unha presión
dada}\label{exemplo-punto-de-burbula-a-unha-presiuxf3n-dada}

Considera o mesmo líquido do problema anterior. Á presión de 5 bar, a
temperatura increméntase gradualmente. Cál é atempertura e a composición
do primeiro vapor que se forma?.

\subsection{Cálculos no punto de
orballo}\label{cuxe1lculos-no-punto-de-orballo}

Neste caso, coñecemos a composición da fase de vapor \(y_i\)
(corresponde ao caso en que \(L\) é moi pequena (\(L \geq 0\)) e
\(y_i = z_i\). O punto de orballo dun vapor (gas) é o punto no que
comeza a condensaar, é dicir, cando se forma a primeira gota de líquido.
A temperatura constante, débese aumentar a presión ata que se forme o
primeiro líquido. A presión constante, débese diminuír a temperatura ata
que se forme o primeiro líquido. En ámbalos dous casos, isto corresponde
axustar \(T\) ou \(p\) ata que

\[
\sum x_i = 1
\]

ou:

\[
\Sigma_{i} y_{i} / K_{i}=1
\]

onde \(y_i\) ven dado. Para unha mestura ideal que cumpre ca \emph{lei
de Raoult}:

\[
\Sigma_{i} \frac{y_{i}}{p_{i}^{\mathrm{sat}}(T)}=\frac{1}{p}
\]

\subsubsection{Caso práctico 01. Punto de burbulla a unha temperatura
dada.}\label{caso-pruxe1ctico-01.-punto-de-burbulla-a-unha-temperatura-dada.}

Unha mestura líquida contén 50\% de moles de pentan, 30\% de moles de
hexano e 20\% de moles de ciclohexano. Calcular a presión do punto de
burbulla cando, á temperatura constante de \(T\) = 400 K, se reduce
progresivamente a presión. Cal é a presión da primeira burbulla que se
fiorma e a composición do primero vapor que se forma supoñendo
comportamento ideal (\textbf{Lei de Raoult}).

Este caso é do tipo 1. O cálculo é inmediato xa que a temperatura é
constante. Simplemente aplicamos as fórmulas. Necesitamos os valores das
constantes de \textbf{ecuación de Antoine} para os compoñentes da
mestura:

\subsubsection{Solución con python}\label{soluciuxf3n-con-python}

Importamos librerías necesarias

\begin{Shaded}
\begin{Highlighting}[]
\ImportTok{import}\NormalTok{ numpy }\ImportTok{as}\NormalTok{ np}
\ImportTok{from}\NormalTok{ scipy.optimize }\ImportTok{import}\NormalTok{ fsolve}
\ImportTok{import}\NormalTok{ matplotlib.pyplot }\ImportTok{as}\NormalTok{ plt }
\end{Highlighting}
\end{Shaded}

\begin{Shaded}
\begin{Highlighting}[]
\CommentTok{\# Datos}
\NormalTok{x\_pentano     }\OperatorTok{=} \FloatTok{0.5}
\NormalTok{x\_hexano      }\OperatorTok{=} \FloatTok{0.3}
\NormalTok{x\_ciclohexano }\OperatorTok{=} \FloatTok{0.2}
\CommentTok{\# pentano C5H12}
\NormalTok{T }\OperatorTok{=} \DecValTok{400} \CommentTok{\# K}
\NormalTok{A\_pentano }\OperatorTok{=} \FloatTok{3.97786}
\NormalTok{B\_pentano }\OperatorTok{=} \FloatTok{1064.840}
\NormalTok{C\_pentano }\OperatorTok{=} \OperatorTok{{-}}\FloatTok{41.136}
\CommentTok{\# hexano C6H14}
\NormalTok{A\_hexano }\OperatorTok{=} \FloatTok{4.00139}
\NormalTok{B\_hexano }\OperatorTok{=} \FloatTok{1170.875}
\NormalTok{C\_hexano }\OperatorTok{=} \OperatorTok{{-}}\FloatTok{48.833}
\CommentTok{\# cyclohexano C6H12}
\NormalTok{A\_ciclohexano }\OperatorTok{=} \FloatTok{3.93002}
\NormalTok{B\_ciclohexano }\OperatorTok{=} \FloatTok{1182.774}
\NormalTok{C\_ciclohexano }\OperatorTok{=} \OperatorTok{{-}}\FloatTok{52.532}
\end{Highlighting}
\end{Shaded}

Despois, calculamos as presións de vapor dos compoñentes puros:

\begin{Shaded}
\begin{Highlighting}[]
\NormalTok{p\_sat\_pentano }\OperatorTok{=}  \DecValTok{10} \OperatorTok{**}\NormalTok{ (A\_pentano }\OperatorTok{{-}}\NormalTok{ B\_pentano }\OperatorTok{/}\NormalTok{ ( C\_pentano }\OperatorTok{+}\NormalTok{ T ))}
\NormalTok{p\_sat\_hexano }\OperatorTok{=}  \DecValTok{10} \OperatorTok{**}\NormalTok{ (A\_hexano }\OperatorTok{{-}}\NormalTok{ B\_hexano }\OperatorTok{/}\NormalTok{ ( C\_hexano }\OperatorTok{+}\NormalTok{ T ))}
\NormalTok{p\_sat\_ciclohexano }\OperatorTok{=}  \DecValTok{10} \OperatorTok{**}\NormalTok{ (A\_ciclohexano }\OperatorTok{{-}}\NormalTok{ B\_ciclohexano }\OperatorTok{/}\NormalTok{ ( C\_ciclohexano }\OperatorTok{+}\NormalTok{ T ))}
\BuiltInTok{print}\NormalTok{(}\StringTok{\textquotesingle{}Presión de saturación do pentano puro    \textquotesingle{}}\NormalTok{, }\StringTok{\textquotesingle{}(\textquotesingle{}}\NormalTok{, T, }\StringTok{\textquotesingle{}) K = \textquotesingle{}}\NormalTok{, p\_sat\_pentano, }\StringTok{\textquotesingle{}bar\textquotesingle{}}\NormalTok{)}
\BuiltInTok{print}\NormalTok{(}\StringTok{\textquotesingle{}Presión de saturación do hexano puro     \textquotesingle{}}\NormalTok{, }\StringTok{\textquotesingle{}(\textquotesingle{}}\NormalTok{, T, }\StringTok{\textquotesingle{}) K = \textquotesingle{}}\NormalTok{, p\_sat\_hexano, }\StringTok{\textquotesingle{}bar\textquotesingle{}}\NormalTok{)}
\BuiltInTok{print}\NormalTok{(}\StringTok{\textquotesingle{}Presión de saturación do ciclohexano puro\textquotesingle{}}\NormalTok{, }\StringTok{\textquotesingle{}(\textquotesingle{}}\NormalTok{, T, }\StringTok{\textquotesingle{}) K = \textquotesingle{}}\NormalTok{, p\_sat\_ciclohexano, }\StringTok{\textquotesingle{}bar\textquotesingle{}}\NormalTok{)}
\end{Highlighting}
\end{Shaded}

\begin{verbatim}
Presión de saturación do pentano puro     ( 400 ) K =  10.247260621669657 bar
Presión de saturación do hexano puro      ( 400 ) K =  4.64675917786267 bar
Presión de saturación do ciclohexano puro ( 400 ) K =  3.3576881112499817 bar
\end{verbatim}

\begin{Shaded}
\begin{Highlighting}[]
\NormalTok{p\_pentano     }\OperatorTok{=}\NormalTok{ x\_pentano     }\OperatorTok{*}\NormalTok{ p\_sat\_pentano}
\NormalTok{p\_hexano      }\OperatorTok{=}\NormalTok{ x\_hexano      }\OperatorTok{*}\NormalTok{ p\_sat\_hexano}
\NormalTok{p\_ciclohexano }\OperatorTok{=}\NormalTok{ x\_ciclohexano }\OperatorTok{*}\NormalTok{ p\_sat\_ciclohexano}
\BuiltInTok{print}\NormalTok{(}\StringTok{\textquotesingle{}Presión parcial do pentano           \textquotesingle{}}\NormalTok{, }\StringTok{\textquotesingle{}(\textquotesingle{}}\NormalTok{, T, }\StringTok{\textquotesingle{}) K = \textquotesingle{}}\NormalTok{, p\_pentano, }\StringTok{\textquotesingle{}bar\textquotesingle{}}\NormalTok{)}
\BuiltInTok{print}\NormalTok{(}\StringTok{\textquotesingle{}Presión parcial do hexano            \textquotesingle{}}\NormalTok{, }\StringTok{\textquotesingle{}(\textquotesingle{}}\NormalTok{, T, }\StringTok{\textquotesingle{}) K = \textquotesingle{}}\NormalTok{, p\_hexano, }\StringTok{\textquotesingle{}bar\textquotesingle{}}\NormalTok{)}
\BuiltInTok{print}\NormalTok{(}\StringTok{\textquotesingle{}Presión parcial do ciclohexano       \textquotesingle{}}\NormalTok{, }\StringTok{\textquotesingle{}(\textquotesingle{}}\NormalTok{, T, }\StringTok{\textquotesingle{}) K = \textquotesingle{}}\NormalTok{, p\_ciclohexano, }\StringTok{\textquotesingle{}bar\textquotesingle{}}\NormalTok{)}
\end{Highlighting}
\end{Shaded}

\begin{verbatim}
Presión parcial do pentano            ( 400 ) K =  5.1236303108348284 bar
Presión parcial do hexano             ( 400 ) K =  1.3940277533588008 bar
Presión parcial do ciclohexano        ( 400 ) K =  0.6715376222499964 bar
\end{verbatim}

A presión total (que é a de orballo) é a suma das presións parciais,

\[
p_{total} = \sum_i p_i
\]

por tanto aó queda sumar as presións parciais e imprimir o resultado:

\begin{Shaded}
\begin{Highlighting}[]
\NormalTok{p\_burbulla    }\OperatorTok{=}\NormalTok{ p\_pentano }\OperatorTok{+}\NormalTok{ p\_hexano }\OperatorTok{+}\NormalTok{ p\_ciclohexano}
\BuiltInTok{print}\NormalTok{(}\StringTok{\textquotesingle{}Presión de burbulla                  \textquotesingle{}}\NormalTok{, }\StringTok{\textquotesingle{}(\textquotesingle{}}\NormalTok{, T, }\StringTok{\textquotesingle{}) K = \textquotesingle{}}\NormalTok{, p\_burbulla, }\StringTok{\textquotesingle{}bar\textquotesingle{}}\NormalTok{)}
\end{Highlighting}
\end{Shaded}

\begin{verbatim}
Presión de burbulla                   ( 400 ) K =  7.189195686443625 bar
\end{verbatim}

As fraccións molares calcúlanse a partires da \textbf{ecuación de
Dalton}:

Como:

\[
p_{total} = \sum_i p_i
\]

e:

\[
p_i = p_{total}*x_i
\]

temos que:

\[
x_i = \frac{p_i}{p_{total}}
\]

\begin{Shaded}
\begin{Highlighting}[]
\NormalTok{x\_pentano     }\OperatorTok{=}\NormalTok{ p\_pentano }\OperatorTok{/}\NormalTok{ p\_burbulla}
\NormalTok{x\_hexano      }\OperatorTok{=}\NormalTok{ p\_hexano }\OperatorTok{/}\NormalTok{ p\_burbulla}
\NormalTok{x\_ciclohexano }\OperatorTok{=}\NormalTok{ p\_ciclohexano }\OperatorTok{/}\NormalTok{ p\_burbulla}
\BuiltInTok{print}\NormalTok{(}\StringTok{\textquotesingle{}Fracción  molar do pentano     no punto de burbulla = \textquotesingle{}}\NormalTok{, x\_pentano)}
\BuiltInTok{print}\NormalTok{(}\StringTok{\textquotesingle{}Fracción  molar do hexano      no punto de burbulla = \textquotesingle{}}\NormalTok{, x\_hexano)}
\BuiltInTok{print}\NormalTok{(}\StringTok{\textquotesingle{}Fracción  molar do ciclohexano no punto de burbulla = \textquotesingle{}}\NormalTok{, x\_ciclohexano)}
\end{Highlighting}
\end{Shaded}

\begin{verbatim}
Fracción  molar do pentano     no punto de burbulla =  0.7126847750849584
Fracción  molar do hexano      no punto de burbulla =  0.19390594082554496
Fracción  molar do ciclohexano no punto de burbulla =  0.09340928408949663
\end{verbatim}

\subsubsection{Caso práctico 02. Punto de burbulla a presión
dada.}\label{caso-pruxe1ctico-02.-punto-de-burbulla-a-presiuxf3n-dada.}

Considera a mesma mestura líquida cun 50\% molar de pentano, un 30\%
molar de hexano (2) e un 20\% molar de ciclohexano (3). A \(p\) = 5 bar,
aumenta gradualmente a temperatura. Cal é a temperatura e a composición
da burbulla do primeiro vapor que se forma?.

\subsubsection{Caso práctico 03. Punto de orballo a temperatura dada
T}\label{caso-pruxe1ctico-03.-punto-de-orballo-a-temperatura-dada-t}

Unha mestura de vapor contén 50\% pentano (1), 30\% hexano (2) e 20\%
ciclohexano (3) (todo en precentaxe molar), é dicir, \(y_1\) = 0,5;
\(y_2\) = 0,3; \(y_3\) = 0,2. A \(T\) = 400 K, a presión aumenta
gradualmente. Cal é a presión do punto de orballo e a composición do
primeiro líquido que se forma?. Asume unha mestura líquida ideal e gas
ideal (Lei de Raoult). Solución. A tarefa é atopar o valor de \(p\): \[
\sum_i \frac {y_i} {p_i^{sat} (T)} = \frac {1} {p}
\]

\subsubsection{Caso práctico 04. Punto de orballo á presión dada
p.}\label{caso-pruxe1ctico-04.-punto-de-orballo-uxe1-presiuxf3n-dada-p.}

Considera un vapor cun 50\% pentano (1), 30\% hexano (2) e 20\%
ciclohexano (3). A \(p\) = 5 bar, a temperatura diminúe gradualmente.
Cal é a temperatura do punto de orballo e a composición do primeiro
líquido que se forma?.

\subsubsection{Caso práctico 05. Punto de orballo con compoñentes non
condensables.}\label{caso-pruxe1ctico-05.-punto-de-orballo-con-compouxf1entes-non-condensables.}

Calcula a temperatura e a composición dun líquido en equilibrio cunha
mestura de gas que conteña 10\% pentano (1), 10\% hexano e 80\%
nitróxeno (3) a 3 bar. O nitróxeno está moi por encima do seu punto
crítico e pode considerarse non condensable.

\subsection{Diagramas T-x-y para mesturas
ideais}\label{diagramas-t-x-y-para-mesturas-ideais}

En ese caso estudiaranse os diagramas e comportamento do equilibrio
vapor-líquido (VLE) dunha mestura \textbf{n-hexano/n-octano}. No
diagrama, a liña azul representa o límite da fase líquida (punto de
burbulla) e a liña verde representa o límite da fase vapor (punto de
orballo).

\subsubsection{Mesturas binarias}\label{mesturas-binarias}

Paraa unha mestura ideal binaria de \(A\) e \(B\), a presión da fase
vapor \(P\) é a suma da presión parcial dos dous compoñentes \(p_A\) e
\(p_B\). A \textbf{Lei de Raoult}, polo contrario, serve só para
mesturas ideais: \[
\begin{align*}
p_A & = x_A P^{sat}_A(T) \\
p_B & = x_B P^{sat}_B(T)
\end{align*}
\] Eliminando as presións partcais e obtendo unha expresión para a
presión de vapor total: \[
P = \underbrace{x_A P_A^{sat}(T)}_{p_A = y_AP} + \underbrace{x_B P_A^{sat}(T)}_{p_B = y_BP}
\] Para mesturas binarias, a substitucións \(x_B = 1-x_A\) e
\(y_B = 1 - y_A\) da unha expresión para a presión total en función da
composición \(x_A\) e a temperatura: \[
P = \underbrace{x_A P_A^{sat}(T)}_{p_A = y_AP}+ \underbrace{(1-x_A) P_A^{sat}(T)}_{p_B=(1-y_B)P}
\] Esta relación pódese demostrar representando \(P\), \(p_A\), e
\(p_B\) como functións de \(x_A\) a temperatura constante.

O seguinte paso é seleccionar unha temperatura \(T\) e evalúar e mostrar
\(P\), \(p_A\) e \(p_B\) fas funcións da fracción molar da fase
líquidaunctions of liquid phase mole fraction \(x_A\).

\subsubsection{Ecuación de Antoine}\label{ecuaciuxf3n-de-antoine}

Os cálculos deste caderno usan a \textbf{Ecuación de Antoine} para
calcular a presión de vapor de saturación para unha temperatura dada e
resolve a \textbf{Ecuación de Antoine} para a temperatura de saturación
dunha presión dada.

A librería \textbf{thermo} implementa un conxunto completo de datos de
propiedades físicas e químicas en Python.

A presión de saturación do compoñente \(i\) calcúlase mediante a
ecuación de \textbf{Antoine}: \[
p_i^{sat} = 10^ \left ( A_i - \frac {B_i} {T+C_i} \right )
\] onde \(i=1\) para o \textbf{n-hexano} e \(i=2\) para o
\textbf{n-octano}, \(p_i^{sat}\) é a presión de saturación (bar),
\(A_i\), \(B_i\) e \(C_i\) son as constantes de \textbf{Antoine}, e
\(T\) a temperatura (°C).

A lei de \textbf{Raoult} úsase para calcular as presións do punto de
burbulla e do punto de orballo por medio do emprego dos factores \(k\):
\[
k_i = \frac {y_i} {x_i} = \frac {p_i^{sat}} {p}
\]

onde \(y_i\) é a fracción molar do compoñente \(i\) na fase vapor, e
\(y_1 + y_2 = 1\), \(x_i\) a fracción molar do compoñente \(i\) na fase
líquida, e \(x_1 + x_2 = 1\), e \(p\) é a presión total (bar).

A presión do punto de burbulla calcúlase usando a seguinte ecuación: \[
\sum k_ix_i =1
\]

\[
p = x_1 p_1^{sat} + x_2 p_2^{sat}
\]

A presión do punto de orballo calcúlase usando a seguinte ecuación: \[
\sum \frac {y_i} {k_i} = 1
\]

\[
p = \left ( \frac {y_1} {p_1^{sat}} + \frac {y_2} {p_2^{sat}} \right )
\]

\subsubsection{Caso práctico 02.}\label{caso-pruxe1ctico-02.}

\begin{enumerate}
\def\labelenumi{\arabic{enumi}.}
\tightlist
\item
  Representa o diagrama P-x-y para unha temperatura de 115 ºC
\item
  Representar o diagrama T-x-y para unha presión de 1.5 bar.
\end{enumerate}

\subsection{Bibliografía}\label{bibliografuxeda}

{[}1{]} J. R. Elliott and C. T. Lira, \emph{Introductory Chemical
Engineering Thermodynamics}, New York: Prentice Hall, 2012 pp.~372--377.



\end{document}
